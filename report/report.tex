\documentclass{article}

\usepackage{graphicx}
\usepackage{natbib}
\usepackage[margin=1in]{geometry}
\usepackage{hyperref}
\usepackage{listings}

\let\cite\citep
\title{A parallel-in-time solver for the heat equation}
\author{Noah Brenowitz and Irena Vankova}
\date{May 18, 2015}
\begin{document}
\maketitle
\section{Introduction}
\label{sec:intro}

The aim of this project is to test the ``Parallel-in-time'' concept
on the heat equation 
\[ u_t = \Delta u \]
with periodic boundary conditions. 
The parallel-in-time strategy is distinct from any sort of spatial
domain decomposition approaches, and we do not implement any spatial
parallelism in this example. 

For Irena: explain parallel in time concept here

here is the citation \cite{falgout2014parallel}. Are there any others
you think we need? What about that ``parallel-in-time'' website?
% Give description of parallel in time approach as described in this paper

\section{Heat equation solver}
\label{sec:ehat}

The serial implementation of the heat equation solver is relatively
straightforward. The typical 2 dimensional discrete laplacian is used
with a implicit trapezoid rule time stepping procedure
(e.g. Crank-Nicolson). The advantage of the Crank-Nicolson scheme is
that it is unconditionally stable like the backward Euler scheme, but
also has second order accuracy.

The solver requires a uniform grid in both
spatial dimensions. In detail, if the discrete laplacian operator is
given by
\[ (L u)_{i,j} = u_{i-1,j} + u_{i+1,j} + u_{i,j-1} + u_{i, j-1} - 4 u_{i,j}, \]
the time step is $\tau$ and the spatial grid size is $h$. Then the
implicit trapezoid rule gives
\[ u^{n+1}  = \frac{\lambda}{2} \left( Lu^n + Lu^{n+1} \right),\]
where the CFL number $\lambda = \tau / h^{2}$.

In practice, this iteration can be written as a matrix
multiplication followed by a linear solve in the following way
\begin{equation}
  \label{eq:time-step}
  u^{n+1} = \left( I- \frac{\lambda}{2} L \right)^{-1} \left(I+\frac{\lambda}{2}L \right) u^n.
\end{equation}


\section{Code design}
\label{sec:code}

The code for this final project is available at \url{https://github.com/nbren12/hpc15final}.
It consists of two parts:
\begin{itemize}
\item The serial heat equation solver provides a library in
  \verb|./src/| called \verb|liblaplace.a| with a simple function that
  evolves the heat equation forward in time with a given time
  step. The signature of this function is
\begin{lstlisting}
void evolve_heat_equation_2d(double *x, int n, double dx,
			     int nt, double dt, int output_interval);
\end{lstlisting}
  
This header provides C-linkage, but the library is written in C++
using a MATLAB like linear algebra package called
\href{http://arma.sourceforge.net/}{Armadillo++} for the sparse linear
solve and matrix multiplications. Armadillo++ is essentially a
high-level wrapper for libraries like BLAS, SuperLU, and LAPACK. The
advantage over using MATLAB or numpy is that there is no temporary
arrays formed with expressions like \verb|a = b + c|.

\item For Irena: explain the ``parallel'' code here.

\end{itemize}

\section{Scalability}
\label{sec:scale}

For Irena: explain the results here


% Bibliography
\bibliographystyle{authordate1}
\bibliography{refs}
\end{document}
